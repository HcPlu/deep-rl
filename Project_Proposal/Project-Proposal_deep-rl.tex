\immediate\write18{makeindex main.nlo -s nomencl.ist -o main.nls}
\documentclass[12pt,oneside]{article}
\usepackage[utf8]{inputenc}
\usepackage[T1]{fontenc}

%----Figures packages
\usepackage{graphicx}
\graphicspath{ {figures/} }
\usepackage{caption}
\usepackage{subcaption}
\usepackage{float}

%----Bibliography  along with 1st line command
\usepackage{nomencl}

%---Math packagesf
\usepackage{amsmath}
\usepackage{amsfonts}
\usepackage{bm}

%---Geometry and style of A4
\usepackage[a4paper,width=150mm,top=5mm,bottom=5mm]{geometry}
\usepackage{fancyhdr}
\pagestyle{fancy}
\renewcommand{\sectionmark}[1]{\markboth{#1}{}} % set the \leftmark
\fancyhf{}

%--Style packages
\renewcommand \thesection{\Roman{section}}
\renewcommand\thesubsection{\Alph{subsection}}
\usepackage{sectsty,textcase}
\subsectionfont{\normalfont\itshape}
\usepackage{titlesec}
\titlelabel{\thetitle.\quad}
\usepackage{titling}


\bibliographystyle{unsrt}

%------Title and Authors
\title{\Huge\textbf{Safe Deep Reinforcement Learning for Multi-Agent System }}
\author{
  Dimitrios Gkouletsos\\
  \texttt{first1.last1@xxxxx.com}
  \and
  Nyquist\\
  \texttt{first2.last2@xxxxx.com}
  \and
  Bode\\
  \texttt{first2.last2@xxxxx.com}
  \and
  Kalman\\
  \texttt{first2.last2@xxxxx.com}
}
\date{\today}

%------------------------BEGIN DOCUMENT-----------------------------------------------------
\begin{document}
\maketitle

%------------------------IDEA-----------------------------------------------------
\section{Idea}
In this project, we propose a Deep Reinforcement Learning algorithm along with safety margins based on \cite{SafeExploration}. Our goal is to extend the 
single-agent application, presented in \cite{SafeExploration}, into a multi-agent enviroment. 

%------------------------METHODS-----------------------------------------------------
\section{Methods}
we use the Deep Deterministic Policy Gradient (DDPG) \cite{DDPG}. 
The constraint is addressed through a linear approximation that utilizes a neural network as well. 

%------------------------DATASETS-----------------------------------------------------
\section{Datasets}
The proposed approach takes advantage of the Open AI gym multi-agent particle environment \cite{environment} in order to apply a set of experiments and 
test the model. 
%------------------------BASELINE-----------------------------------------------------
\section{Baseline}
%We must write how to compare

%------------------------GOALS-----------------------------------------------------
\section{Goals}
%What results do we want ? How are these connected to the Deep Learning Course ? 

%------------------------LITERATURE OVERVIEW-----------------------------------------------------
\section{Literature Overview}

%-----------------------REFERENCES----------------------------------------------------
\bibliography{references}

\end{document}